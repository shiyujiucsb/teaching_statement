\documentclass[12pt]{article}

\usepackage{fullpage}

\begin{document}
\title{Teaching Statment}
\author{Shiyu Ji}\date{}
\maketitle

One of the most enjoyable experiences in academic life is to communicate with other smart people. Teaching offers such an excellent platform. Below I give my detailed previous teaching experience and my teaching philosolphy.

{\bf Previous Experience:} When I was with Oklahoma State University, I had the privilege of being a teaching assistant (TA) for many theoretical undergraduate courses, such as Theory of Computation, Algorithms and Data Structures, Cryptography, etc. My main task is to homework/exam grading and answering the questions from the students in person or by emails. For Algorithms and Data Structures, I also gave 7 lectures on heap sort, hashing and binary search trees. For Cryptography, I introduced crypto++, a well known cryptographic library to the students and helped them finish the assignments. 
After I came to UCSB, I was a TA for CS 40, Foundations of Computer Science. In addition to grading the homeworks and exams, I also lead discussions in the sections, answering students and proposing interesting questions to intrigue discussions with the students. Since most courses of my TAing are theoretical, I need to spend a lot of time explaining the details to help students better understand. According to the feedbacks, many students found they cound better understand the materials with my clarifications. For example, many students found it difficult to understand Heap Sort. To help them, I went through a concrete example to sort a number sequence on the blackboard, and presented an animation on the screen using a projector. After the class, the students thought many doubts in their mind had been completely resolved. 

When I graded the homeworks and exams, I always gave comments where the students did wrongly and thus lost credits. I believe this is another great way to boost communication with the students. Sometimes texts can 

{\bf Teaching Philosophy:} I totally agree with the credo that teaching and learning should be interesting inherently, even though the teaching materials can be highly abstract and theoretical. I believe that every conception has its own background story, which motivated the rise of the conception in history. Hence the students should be interested in the story and conception, if the teacher can tell the story clearly. For example, it is usually supposed to be difficult to understand the truth values of the implication. However, it proves easy to help the students understand them by examples that are interesting to the students, such like the statement ``each UCSB undergraduate student must have high school GPA larger than 3.0''. That is, a student whose GPA is larger than 3.0 is not necessarily with UCSB, but usually a student whose GPA is less than 3.0 cannot get admitted by UCSB. Most students found such an example is familiar and interesting to them. Discussing the examples can be definitely beneficial to help understand and memorize the truth values of the implication. Every everlasting theory comes from the real world. The fun and desire to explore the real world keep driving the theoretical research in computer science. Teaching should always reflect this fact.

{\bf Future Plan:} While keeping my basic teachin philosophy, I will try to further improve my teaching skills by leading sections, discussing with the students in my office hours, learning presentation skills by attending lectures and talks. To achieve this, I must aim at higher platform to release my potential as much as possible and give better performance.

\end{document}
